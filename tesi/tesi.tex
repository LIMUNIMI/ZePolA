%%%%%%%%%%%%%%%%%%%%%%%%%%%%%%%%%%%%%%%%%%%%%%%%%%%%%%%%%%%%%%%%%%%%%%%%%%%%%%%
%                                                                             %
%           TEMPLATE LATEX PER TESI                                           %
%           ______________                                                    %
%                                                                             %
%           Ultima revisione: 09 Ottobre 2024                                  %
%           Revisori: G. Presti; L. A. Ludovico; F. Avanzini; M. Tiraboschi   %
%                                                                             %
%%%%%%%%%%%%%%%%%%%%%%%%%%%%%%%%%%%%%%%%%%%%%%%%%%%%%%%%%%%%%%%%%%%%%%%%%%%%%%%

\documentclass[12pt]{report}

% --- PREAMBOLO ---------------------------------------------------------------
% Inserire qui eventuali package da includere o 
% definizioni di comandi personalizzati

% Selezione lingua
\usepackage[italian]{babel}

\usepackage{tesi}
% Puoi usare il font di default di LaTeX con la relativa opzione del package
% \usepackage[defaultfont]{tesi}
% Esiste anche un'opzione per il formato 17x24 per le tesi di dottorato
% \usepackage[phd]{tesi}

% In caso il copia-incolla del PDF generato perda gli spazi,
% provare a decommentare la seguente riga
% \pdfinterwordspaceon

% !!! INFORMAZIONI SULLA TESI DA COMPILARE !!!

%   UNIVERSITA' E CORSO DI LAUREA:
\university{Università degli Studi di Milano}
\unilogo{immagini/unimi}
\faculty{Facoltà di Scienze e Tecnologie}
\department{Dipartimento di Informatica\\Giovanni Degli Antoni}
\cdl{Corso di Laurea Triennale in Informatica Musicale}

%   TITOLO TESI:
\title{Equalizzatore parametrico con GUI per poli e zeri: uno strumento didattico}
% Questo comando (opzionale) sovrascrive \title per quanto riguarda la copertina
% Può essere usato per stampare caratteri speciali, tenendo i metadati puliti
\printedtitle{Equalizzatore parametrico con GUI per poli e zeri: uno strumento didattico}

%   AUTORE:
\author{Andrea Casati}
\matricola{963119}
% "Elaborato Finale" per i CdL triennali
% "Tesi di Laurea" per i CdL magistrali
\typeofthesis{Elaborato Finale}

%   RELATORE E CORRELATORE:
\relatore{Prof. Giorgio Presti}
\correlatore{Prof. Marco Tiraboschi}

%   LABORATORIO:
% Questa sezione crea una pagina di chiusura della tesi con
% il logo dell'ente/laboratorio presso cui si è svolto il tirocinio.
% Più afferenze/url/loghi sono supportate,
% e la frase può essere personalizzata.
% Qui trovate alcuni predefiniti del nostro dipartimento
% \adaptlab
% \aislab
% \anacletolab
% \bisplab
% \connetslab
% \everywarelab
% \falselab
% \iebilab
% \islab
% \lailalab
% \lalalab
% \lawlab
% \laserlab
% \limlab
% \mipslab
% \optlab
% \phuselab
% \ponglab
% \sesarlab
% \spdplab

% Esempio di personalizzazione della pagina di chiusura
% (non consegnate con questo esempio!)
% (da commentare in caso sia sufficiente una delle macro precedenti)
% \lab{Laboratorio di Ricerca}
% \lab[in collaborazione con l']{Azienda Specifica}
% \laburl{https://di.unimi.it/it/ricerca/risorse-e-luoghi-della-ricerca/laboratori-di-ricerca}
% \lablogo{immagini/redqmark}

%   ANNO ACCADEMICO:
% \the\year inserisce l'anno corrente
% per specificare manualmente un anno accademico
% NON inserire nel formato 1970-1971, ma
% inserire solo 1970
\academicyear{\the\year} 

%   INDICI:
% elenco delle figure (facoltativo)
% \figurespagetrue
% elenco delle tabelle (facoltativo)
% \tablespagetrue
% prefazioni nell'indice (facoltativo)
\prefaceintoctrue
% indice nell'indice (facoltativo)
\tocintoctrue
% --- FINE PREAMBOLO ----------------------------------------------------------

\begin{document}

% Creazione automatica della copertina
% Centra la copertina nel foglio: usa questo comando per la copertina esterna
% \makecenteredfrontpage
% Copertina allineata alle altre pagine: usa questo comando per la copertina interna
\makefrontpage

% 
%			PAGINA DI DEDICA E/O CITAZIONE
%			facoltativa, questa è l'unica cosa che dovete formattare a mano, un po' come vi pare
%

{\raggedleft \large \sl Questo lavoro \`{e} dedicato ai miei genitori\\}

\clearpage
\beforepreface

% 
%			PREFAZIONE (facoltativa)
%

% \prefacesection{Prefazione}
% Le prefazioni non sono molto comuni, tuttavia a volte capita che qualcuno voglia dire qualcosa che esuli dal lavoro in sé (come un meta-commento sull'elaborato), o voglia fornire informazioni riguardanti l'eventuale progetto entro cui la tesi si colloca (in questo caso è probabile che sia il relatore a scrivere questa parte).

%
%			RINGRAZIAMENTI (facoltativi)
%

\prefacesection{Ringraziamenti}
Questa sezione, facoltativa, contiene i ringraziamenti.

%
%			Creazione automatica dell'indice
%

\afterpreface

% 
%			CAPITOLO 1: Introduzione o Abstract
% 

\chapter{Introduzione}
\label{cap:introduzione}

\section{Il template}

\section{I contenuti}

\section{Organizzazione della tesi}

\section{Stile e forma}

% 
%			CAPITOLO 2: Stato dell'arte
% 

\chapter{Stato dell'arte}
\label{chap:stato_arte}
La crescita della ricerca e sviluppo nel filtraggio digitale è iniziata negli anni Sessanta, quando l'uso del filtraggio digitale ricorsivo ha aperto la strada alla simulazione di filtri analogici nel dominio digitale. 

I primi studi sul filtraggio digitale sono stati incentivati dalla necessità di creare tecnologie capaci di simulare dispositivi per la modifica della larghezza di banda della voce. L'uso dei filtri analogici per il trattamento dei segnali vocali, infatti, comportava varie difficoltà, tra cui costi elevati, peso e ingombro dell'hardware.

I primi approcci al filtraggio digitale, in particolare al filtraggio ricorsivo, si sono basati sul concetto di convoluzione: l’output di un filtro digitale è legato al suo input tramite l’integrale di convoluzione, approssimabile in digitale come somma di convoluzione del segnale d’ingresso e della risposta all’impulso campionati. La principale difficoltà, oltre alla limitata capacità di calcolo degli anni Sessanta rispetto a oggi, era la gestione della risposta all'impulso infinita dei filtri ricorsivi.

L’intuizione fu quella di applicare lo stesso principio utilizzato nei filtri analogici: l’uso di una \textit{memoria}. I filtri analogici, infatti, generano ricorsività attraverso circuiti di retroazione, in cui l’uscita viene riproposta in ingresso.

Lo studio delle equazioni alle differenze nei filtri analogici rese quindi naturale l’applicazione delle stesse in ambito digitale, tramite l'utilizzo della trasformata Z, permettendo di implementare queste equazioni come un vero e proprio software applicativo.

I passaggi successivi riguardarono la mappatura del piano \( s \) (tramite la trasformata di Laplace) nel piano \( z \) (tramite la trasformata \( Z \)) per il calcolo di poli e zeri del filtro e la loro effettiva implementazione. Se nei filtri analogici poli e zeri vengono realizzati attraverso l’uso di resistori, condensatori e induttori, in digitale si è subito dimostrato possibile realizzarli esclusivamente mediante calcoli matematici, con particolare attenzione alla precisione numerica.


Fin dall'inizio, l'elaborazione numerica digitale dei segnali ha mostrato vantaggi rispetto alla controparte analogica: mentre l’elaborazione analogica può dimostrarsi effimera, quella digitale permette il salvataggio dell'ingresso, dello stato di un filtro e dell'uscita.

Nel corso degli anni, in particolare dagli anni Settanta (periodo delle prime applicazioni pratiche dei filtri digitali) fino a oggi, le proprietà e gli utilizzi dei filtri digitali non ricorsivi FIR (con risposta all’impulso finita) e ricorsivi IIR (con risposta all’impulso infinita) sono cambiati significativamente. Nella decade 1970-1980, l’applicazione principale dei filtri digitali è stata il filtraggio selettivo, contesto in cui l’uso dei filtri digitali ricorsivi ha dimostrato di essere più vantaggioso.

Negli anni successivi, i filtri digitali sono stati impiegati anche per altri scopi, oltre al filtraggio selettivo, come il filtraggio adattativo e il filtraggio adattato (utilizzato nelle telecomunicazioni). In questi contesti, i filtri FIR si sono rivelati particolarmente efficaci. Durante gli anni Ottanta, quindi, i filtri digitali FIR sono diventati preferibili per questi diversi utilizzi.

Con il progresso degli anni Novanta e la possibilità di rendere stabili i filtri digitali IIR e a fase lineare, questi ultimi hanno ritrovato un ruolo importante nelle applicazioni digitali.

Oggi i filtri digitali ricorsivi sono considerati un metodo efficace per approssimare una risposta all’impulso infinita. In teoria, la risposta all’impulso di un filtro ricorsivo è infinita nel tempo, ma in pratica essa decresce esponenzialmente verso lo zero. Nel dominio digitale, infatti, l’errore di approssimazione diventa trascurabile già dopo poche centinaia di campioni.

\section{Equalizzazione audio digitale}
\subsection{Equalizzazione nell'elaborazione di segnali}
La manipolazione dei segnali analogici e digitali ha origini che risalgono agli inizi della produzione audio e di altre applicazioni, e in questo contesto l’equalizzazione, sia in ambito analogico che digitale, rappresenta un passaggio fondamentale.
Il termine equalizzazione ha origine nell'ingegneria telefonica dei primi tempi, settore in cui era necessario correggere i segnali telefonici degradati, soprattutto a causa della perdita delle alte frequenze nei lunghi tragitti, per ripristinare il segnale originale all'arrivo a destinazione.
L'equalizzazione è quindi un processo cruciale nell'elaborazione audio: consente infatti di enfatizzare o attenuare specifiche bande di frequenza, migliorando la chiarezza, l'intelligibilità o il carattere sonoro di un segnale audio. Dagli studi di registrazione alle applicazioni di live sound e produzione musicale, gli equalizzatori digitali sono utilizzati per adattare il suono a contesti acustici diversi, correggere problematiche sonore o migliorare il contenuto audio in base ai gusti estetici e alle esigenze del mixaggio.

Con l’avvento dei filtri digitali, l’equalizzazione in questo dominio ha raggiunto un alto livello di flessibilità, permettendo sia la riproduzione e simulazione di strumenti analogici, sia la creazione di strumenti e software innovativi che superano le 
più classiche e tradizionali implementazioni.

Gli equalizzatori digitali si dividono in categorie:
\begin{itemize}
    \item Equalizzatori grafici: strumenti per la manipolazione della risposta in frequenza, che offrono un controllo di volume (attenuazione e guadagno) su bande di frequenza specifiche, stabilite in fase di progettazione (solitamente rispettando le frequenze presenti nelle tabelle standard ISO). Ogni banda è regolabile tramite cursori o controlli dedicati, che agiscono su filtri passa-banda per modellare lo spettro audio secondo le preferenze dell'utente.
    \item Equalizzatori parametrici: strumenti per la manipolazione della risposta in frequenza che offrono un controllo adattabile su ogni banda di frequenza. Consentono di regolare non solo il guadagno o l'attenuazione, ma anche la frequenza centrale e la larghezza di banda (o Q factor) di ogni singolo filtro. Questi parametri permettono una modellazione precisa dello spettro audio, adattandosi meglio alle esigenze specifiche dell'utente.
    \item Equalizzatori dinamici: strumenti per la manipolazione della risposta in frequenza che offrono un controllo basato sul principio degli equalizzatori parametrici. Tuttavia, a differenza di questi ultimi, l'attenuazione o il guadagno di specifiche bande dipende dal superamento di una soglia di volume nello spettro della stessa banda di frequenza. La risposta del filtro cambia quindi a seconda della variazione dinamica dello spettro del segnale.
\end{itemize}

\subsection{Metodi di progettazione di equalizzatori digitali}
Gran parte delle applicazioni di elaborazione dei segnali riguarda l'attenuazione o l'accentuazione di una porzione specifica dello spettro del segnale. In ambito audio, questo obiettivo viene spesso raggiunto attraverso l’uso di filtri biquadratici, che consentono di attenuare o amplificare una specifica banda di frequenze attorno a una frequenza centrale. Nel dominio digitale, è possibile ottenere qualsiasi risposta in frequenza tramite una cascata di filtri biquadratici disposti in serie.



\section{Poli e zeri nei filtri digitali}

\section{Prototipi di filtri analogici per il design di filtri digitali}
BIBLIOGRAFIA: Digital filter design techniques in the frequency domain

\section{Tecnologie per il design interattivo di filtri}

\section{Strumenti e tecniche educative per il DSP audio}

\section{Sintesi e gap da colmare}


% 
%			CAPITOLO 3: Tecnologie utilizzate
%			(omettere questo capitolo se non necessario)
% 

\chapter{Tecnologie utilizzate}
\label{cap3}

In questo capitolo vengono presentati alcuni suggerimenti utili per un utente \LaTeX\ alle prime armi.


\section{Generalit\`a}

\subsection{La scrittura WYSIWYG vs.\ WYSIWYM}

L'acronimo WYSIWYG sta per ``What You See is What You Get'', e si riferisce al concetto di ottenere sulla carta testo e immagini che abbiano una disposizione grafica equivalente a quella visualizzata a schermo dal software di videoscrittura. Un esempio classico di WYSIWYG è Microsoft Word, che mostra il testo impaginato e formattato come ci si aspetta di vederlo una volta stampato.

L'acronimo WYSIWYM sta per ``What You See is What You Mean'', ed è il paradigma per la creazione di testi strutturati. \LaTeX\ è un ambiente che supporta tale paradigma. In realtà, anche Microsoft Word avrebbe la possibilità di strutturare il testo, principalmente attraverso il meccanismo degli stili, ma pochissimi utenti sfruttano tale funzionalità (ovviamente se sceglierete di scrivere la tesi in Word raccomandiamo caldamente l'uso di tali funzioni).

I principali svantaggi di un sistema WYSIWYM sono il tempo di apprendimento, dovuto a una minore intuitività degli strumenti software, e la necessità di invocare la compilazione del documento per vederne l'aspetto definitivo. Ad esempio, in \LaTeX\ l'intero documento viene scritto in testo semplice, che all'interno contiene ambienti e comandi con informazioni di layout, e solo la compilazione permette di scoprire eventuali errori di sintassi e giungere, infine, alla creazione del PDF. 

Le difficoltà iniziali, però, sono ampiamente compensate dai vantaggi a medio e lungo termine. Infatti, il lavoro risulterà perfettamente impaginato e strutturato, e dunque avrà un aspetto professionale. Questo riguarda non solo gli stili, che vengono applicati al testo in modo coerente con il template prescelto, ma anche problemi tipicamente spinosi di Word, quali il posizionamento delle immagini e delle tabelle, la creazione di una bibliografia con relative citazioni nel testo, la creazione di un sommario (per cui esistono funzioni automatiche, ma sono piuttosto macchinose). Diventa automatico e molto semplice, ad esempio, aggiungere un indice delle figure o delle tabelle, oppure numerare le formule espresse nel testo. Un altro aspetto su cui \LaTeX\ è nettamente superiore a Word è proprio la scrittura di formule matematiche, come mostrato nell'esempio qui riportato:
\begin{equation}
x_i(n) = a_{i1}u_1(n) + a_{i2}u_2(n) + \cdots + a_{iJ}u_J(n) \, .
\label{eq:multimix}
\end{equation}

\subsection{Risorse e strumenti}

Esiste una vastissima gamma di risorse online per avvicinarsi a \LaTeX. Un buon punto di partenza \`e la lista messa a disposizione sul sito del \TeX\ Users Group (TUG).\footnote{\url{http://www.tug.org/interest.html}}
Tra queste si consiglia in particolare la ``Not so Short Introduction to LaTeX2e'',\footnote{\url{http://mirrors.ibiblio.org/CTAN/info/lshort/}}. Per chi volesse approfondire, uno dei riferimenti bibliografici pi\`u completi \`e il libro di Mittelbach {\em et al.}~\cite{mittelbach2004latex}.

In alternativa a un'installazione locale sul proprio pc, \`e possibile utilizzare un editor \LaTeX\ online, con il vantaggio di avere immediatamente a disposizione l'ambiente di sviluppo e tutti i package necessari, nonch\'e di potere condividere il proprio progetto con il relatore di tesi. Il pi\`u diffuso editor \LaTeX\ online \`e Overleaf,\footnote{\url{http://www.overleaf.com}} dove si pu\`o trovare anche ulteriore documentazione (in particolare la guida ``Learn \LaTeX\ in 30 minutes'').\footnote{\url{https://www.overleaf.com/learn}}

Qualunque sia la risorsa utilizzata, ecco un elenco non esaustivo di argomenti di base nei quali con tutta probabilit\`a ci si imbatter\`a durante la stesura della tesi.
\begin{itemize}
\item Formattazione del testo (grassetto, italics, dimensioni font, ecc.) e del documento (paragrafi, comandi \verb|\chapter|, \verb|\section|, \verb|\tableofcontents|, ecc.).
\item Elenchi: ambienti {\em itemize} e {\em enumerate}, pacchetti rilevanti ({\em paralist})
\item Riferimenti incrociati: comandi \verb|\ref|, \verb|\pageref| e \verb|\label|, etichette.
\item Matematica: equazioni, modalit\`a {\em inline} e {\em displayed}, pacchetti rilevanti ({\em amssymb}, {\em amsmath}).
\item Figure: formati grafici, ambiente {\em figure}, pacchetti rilevanti ({\em graphicx}, {\em subfloats} per figure multiple).
\item Tabelle: ambienti {\em table} e {\em tabular}, pacchetti rilevanti ({\em array}, {\em multirow}, {\em longtable}).
\item Riferimenti e bibliografie (si veda pi\`u sotto la sezione~\ref{sec:bibtex}).
\end{itemize}

\section{Suggerimenti sull'uso di \LaTeX}
\label{sec:consigli_latex}

Fatte salve le indicazioni generali fornite nella sezione precedente, di seguito si riportano alcune osservazioni puntuali sulle domande e gli errori pi\`u tipici degli studenti alle prime armi con \LaTeX.

\subsection{Riferimenti incrociati}

Uno dei principali vantaggi di \LaTeX\ è la possibilità di impostare riferimenti automatici a molti elementi del documento, tra cui capitoli, sezioni, sottosezioni, tabelle, figure, equazioni, riferimenti bibliografici, e via dicendo.

Quindi il modo corretto per riferirsi, ad esempio, al secondo capitolo non è scrivere ``Capitolo 2'' bensì ``Capitolo~\ref{chap:stato_arte}''. Il risultato apparente (nel PDF) è lo stesso, mentre ci sono differenze sostanziali a livello di codice. Il vantaggio è che, se il secondo capitolo diventasse il terzo, il riferimento incrociato continuerebbe a puntare alla posizione corretta. Si pensi, per estensione, alla numerazione delle immagini, o ai riferimenti alla bibliografia.

Sintatticamente, questo richiede di inserire dei comandi \verb|\label{mia_label}| all'interno degli elementi cui ci si vuole riferire, e dei comandi \verb|\ref{mia_label}| dove si vuole creare il riferimento. Fa eccezione la bibliografia (si veda pi\`u sotto la sezione~\ref{sec:bibtex}).


\subsection{Ritorni a capo}

I ritorni a capo in \LaTeX\ possono essere effettuati in due modi: con la sintassi \verb|\\| o con una doppia pressione del tasto di ritorno a capo. In generale, la soluzione corretta è la seconda, che equivale a usare il tasto Enter in Word. Il doppio Backslash, che corrisponde a Shift+Enter in Word, crea una nuova riga senza interruzione del paragrafo. Questo va usato solo in casi molto specifici, come nella frase seguente.

Il sito web ufficiale dell'Università degli Studi di Milano è:\\
\url{https://www.unimi.it}.

In molti stili di \LaTeX, un nuovo paragrafo (dopo un doppio a capo) crea un rientro della prima riga. Non c'\`e nulla di male nel rientro, ma se proprio lo si vuole evitare la soluzione \textbf{non} \`e usare il doppio BackSlash! Esistono molte soluzioni pi\`u appropriate (ad esempio, dare una dimensione nulla al rientro tramite il comando \verb|\setlength{\parindent}{0ex}|, da inserire nel preambolo della tesi).

\subsection{Accenti}

Scrivendo la tesi in italiano, l'uso di lettere accentate \`e frequente. I caratteri accentati immessi da tastiera non vengono per\`o riconosciuti nativamente. Invece l'accento grave e acuto in \LaTeX\ si ottengono rispettivamente con i comandi \verb|\`{a}| e \verb|\'{a}.| 

In alternativa \`e possibile specificare che si usa la codifica UTF-8, usando il comando \verb|\usepackage[utf8]{inputenc}| nel preambolo del documento (già incluso in questo template). In questo modo i caratteri accentati immessi da tastiera verranno riconosciuti.

Nota ortografica: attenzione a non sbagliare gli accenti: si scrive ``\`e'', ma si scrive ``perch\'e''.

\subsection{Spazi tra parole}

Riguardo la gestione della spaziatura tra parole, \LaTeX\ adotta una strategia molto elegante, che lascia uno spazio maggiorato dopo il punto di fine periodo. Un potenziale problema è che questo spazio extra viene introdotto dopo qualsiasi occorrenza del punto, indipendentemente dalla funzione sintattica, e dunque anche dopo i nomi puntati, quali ``R. Schumann'', o dopo le formule ``ad es.'', ``Fig. n'', ``ecc.'' e via dicendo. Per evitarlo, questi spazi da non aumentare vanno sostituiti con alternative, quali un Backslash seguito da uno spazio (che immette un \textit{control space}) o una tilde \verb|~| (che introduce un \textit{unbreakable space}, utile a impedire ritorni a capo intermedi).\footnote{Per una trattazione completa delle numerose varianti, si veda \url{https://tex.stackexchange.com/questions/74353/what-commands-are-there-for-horizontal-spacing}}

\subsection{Interlinea}

Per aumentare la leggibilit\`a della tesi pu\`o essere utile usare un'interlinea maggiore di 1. Un modo per ottenerlo \`e usare il comando \verb|\linespread{1.6}| nel preambolo del documento. Nota: il valore $1.6$ indica interlinea doppia, il valore $1.3$ indica interlinea 1.5. Don'ask why.


\subsection{Doppie virgolette}

L'uso dell'unico carattere di doppie virgolette presente sulla tastiera è assolutamente sconsigliato, in quanto non viene correttamente interpretato da \LaTeX, soprattutto riguardo l'apertura delle virgolette.

La combinazione giusta da utilizzare è \verb|``| per l'apertura e \verb|''| per la chiusura. Si noti che in entrambi i casi si tratta di doppi apostrofi ravvicinati, e non di singoli caratteri. Se si utilizza come editor TeXstudio, c'è un'opzione per sostituire automaticamente le doppie virgolette con la versione corretta in \LaTeX: Opzioni $\rightarrow$ Configura TeXstudio... $\rightarrow$ Editore $\rightarrow$ Sostituisci i doppi apici: Inglesi.

Le virgolette caporali, o francesi, si ottengono con i comandi \verb|\guillemotleft| e \break\verb|\guillemotright|, il cui risultato è \guillemotleft questo\guillemotright.


\subsection{Ambienti per scrivere codice}

Il codice all'interno dell'elaborato va scritto con carattere monospaziato e rispettando, nell'ambito del possibile, le originali regole (o buone pratiche) di indentazione.

Per farlo, esiste innanzi tutto l'ambiente verbatim, che va aperto e chiuso con i comandi \verb|\begin{verbatim}| ed \verb|\end{verbatim}|.

Tra le alternative, si segnala l'ambiente lstlisting, che richiede innanzi tutto di aggiungere nel preambolo \verb|\usepackage{listings}|, e quindi di aprire e chiudere l'ambiente con i comandi \verb|\begin{lstlisting}| ed \verb|\end{lstlisting}|. Un esempio, relativo al calcolo del massimo comun divisore attraverso l'algoritmo di Euclide in Python, è:

\begin{lstlisting}
def MCD(a,b):
	while b != 0:
		a, b = b, a % b
	return a
\end{lstlisting}

Se dopo l'apertura dell'ambiente si specifica tra parentesi quadrate il linguaggio adottato, ad esempio con la sintassi \verb|\begin{lstlisting}[language=Python]|, si ottiene automaticamente l'evidenziazione del codice:

\begin{lstlisting}[language=Python]
def MCD(a,b):
	while b != 0:
		a, b = b, a % b
	return a
\end{lstlisting}

L'elenco dei linguaggi supportati e le opzioni avanzate per personalizzare la visualizzazione del codice si trovano all'indirizzo \url{https://www.overleaf.com/learn/latex/Code_listing#Reference_guide}.

Si consideri anche la possibilità di importare interi file di codice attraverso la sintassi \verb|\lstinputlisting[language=nomelinguaggio]{filesorgente}|.

In fine, se lstlisting non dovesse incontrare il vostro gusto, si segnala che in alternativa è possibile usare il package minted.

\subsection{Figure}

In tutti i casi in cui sia possibile (schemi a blocchi, plot di dati, ecc.), \`e opportuno che le figure siano in formato vettoriale (eps, pdf) per aumentarne la leggibilit\`a.
Nel caso di figure prodotte da software esterno (ad esempio, grafici esportati in eps o pdf da Matlab), \`e consigliabile conservare tutti i sorgenti e i dati utilizzati per generarle: in questo modo sarà possibile ricreare le figure quando necessario.

Le figure devono sempre avere riferimenti nel testo, realizzati assegnando un'etichetta alla figura mediante il comando \verb|\label{nomeEtichetta}| subito prima della chiusura dell'ambiente \verb|figure|, e utilizzandola nel testo mediante il comando \verb|\ref{nomeEtichetta}|. Il medesimo discorso vale anche per Tabelle ed Equazioni.

Si evitino espressioni del tipo ``come visibile nella figura seguente'' in favore di riferimenti esatti del tipo ``come visibile in Fig~\ref{fig:ideas2text}'' in quanto \LaTeX posiziona le immagini sulla pagina seguendo regole tipografiche, che non corrispondono necessariamente alla posizione di inserimento nel sorgente del documento.

\subsection{Commenti e revisione}

Overleaf mette a disposizione delle funzionalità di revisione, come ad esempio la possibilità di inserire commenti relativi a punti specifici del sorgente, senza che questi intacchino il sorgente stesso o il file compilato. Alcuni professori preferiscono usare questo strumento, altri invece preferiscono lasciare traccia dei commenti direttamente nel codice, in modo che vengano versionati insieme al sorgente e compaiano sul PDF compilato.

In caso si preferisca commentare il codice nel sorgente invece che usare le funzioni di review di Overleaf è necessario aumentare i margini del documento per permettere alle note di stare accanto al corpo del testo, e successivamente importare il package todonotes. Per fare ciò è necessario aggiungere nel preambolo i seguenti comandi (su due righe ed in questo ordine) \verb|\setlength {\marginparwidth }{2cm}| e \verb|\usepackage{todonotes}|. A tesi completata è possibile nascondere i commenti senza doverli eliminare manualmente modificando l'inclusione del package come segue: \verb|\usepackage[disable]{todonotes}|.


\section{\hologo{BibTeX}}
\label{sec:bibtex}

\subsection{Generalit\`a}

Esistono più modi per inserire una bibliografia in \LaTeX. Si consiglia fortemente l'utilizzo del sistema \hologo{BibTeX}. Questo consente di aggiungere, rimuovere e modificare voci di bibliografia in maniera efficiente, di formattarle, di riordinarle a piacere e aggiornare automaticamente i corrispondenti riferimenti nel testo, ecc.

Una guida introduttiva e completa \`e ``Tame the BeaST''.\footnote{Accessibile da \url{http://www.tug.org/interest.html}} 
In estrema sintesi, i passi per gestire una bibliografia tramite \hologo{BibTeX} sono essenzialmente tre.
\begin{enumerate}
\item Salvare i riferimenti bibliografici come entry di uno o pi\`u file con l'estensione .bib (si veda ad esempio il file \texttt{bibliografia.bib}, parte di questo template). Gli entry sono scritti in un formato specifico, in particolare ogni entry ha una propria etichetta testuale che lo identifica univocamente.
\item Creare la bibliografia alla fine del documento o dove desiderato, usando il comando \verb|\bibliography{file1.bib,file2.bib,...}|. \`E possibile inoltre specificare uno stile bibliografico attraverso il comando \verb|\biblographystyle{...}|.
\item All'interno del testo, riferirsi a una voce di bibliografia tramite il comando\\ \verb|\cite{etichetta_entry}|. Si noti che una voce bibliografica non viene inclusa in bibliografia in assenza di una citazione all'interno del testo (coerentemente con quanto discusso nella sezione~\ref{sec:biblio}).
\end{enumerate}

\`E consigliabile cominciare a costruire la propria bibliografia in formato bib a mano a mano che si analizza lo stato dell'arte, invece che rimandare alla stesura finale della tesi.

\subsection{Strumenti}

Un file .bib \`e un file di testo e pu\`o quindi essere gestito con un qualsiasi text editor. Esistono comunque molti tool pi\`u evoluti per gestire bibliografie in formato bib. Un'applicazione installabile localmente sul proprio pc \`e JabRef.\footnote{\url{http://www.jabref.org}}. Oppure esistono tool online, come Zotero,\footnote{\url{http://www.zotero.org}} che forniscono molte funzionalità tra cui l'esportazione di bibliografie in formato bib.

Peraltro, anche Google Scholar esporta automaticamente citazioni in formato bib cliccando sul link Cita (icona con doppie virgolette) e scegliendo l'opzione \hologo{BibTeX} nella parte bassa della finestra che si apre. \textbf{Attenzione} per\`o: spesso i bib esportati da Scholar sono incompleti o sporchi, \`e sempre consigliabile controllarne la correttezza.

Infatti, si preferiscono, in generale, i metadati raccolti dalla sorgente primaria della risorsa (es. sito ufficiale della pubblicazione) rispetto a quelli presentati dai motori di ricerca. Nel caso in cui i siti ufficiali non propongano il formato \hologo{BibTeX}, esistono convertitori, facilmente reperibili e utilizzabili via web, per convertire in \hologo{BibTeX} la maggior parte dei formati (es. RIS). Nei rari casi in cui non siano disponibili i dati in alcun formato machine readable, potete compilare manualmente l'entry \hologo{BibTeX} voi stessi. Per un riferimento su tipi di entry e i relativi campi, potete consultare \url{https://bibtex.eu}.

I pi\`u ``smanettoni'' possono addentrarsi a piacere in funzionalit\`a avanzate. Ad esempio \`e possibile creare bibliografie multiple.\footnote{Un possibile modo viene illustrato qui: \url{https://bit.ly/2wI3Y7p}}
Esiste anche il pacchetto \texttt{biblatex}, che fornisce una reimplementazione completa delle funzionalit\`a bibliografiche di \LaTeX-\hologo{BibTeX}, offrendo maggiore flessibilit\`a e mantenendo compatibilit\`a all'indietro con il formato bib.





% 
%			CAPITOLO 4: Il lavoro svolto
% 

\chapter{Nome del Progetto}
\label{cap4}

In genere il capitolo più corposo dell'elaborato è quello in cui si parla del lavoro svolto. Esistono alcune buone pratiche per rendere l'esposizione efficace, eccone alcune.

\section{Panoramica del progetto}

Prima di addentrarsi nei dettagli è bene fornire una panoramica (anche molto schematica, corredata da un diagramma) del lavoro svolto, in modo che il lettore abbia una mappa concettuale con cui orientarsi.

\section{Implementazione}

Una volta fornita una panoramica, è possibile addentrarsi nei dettagli, ricordando che si sta scrivendo un articolo scientifico e non un testo di narrativa. Vanno dunque evitate domande retoriche quali ``\textit{ma come è stato possibile risolvere questo problema? Ebbene \ldots}''.

Vale la pena riportare nel testo solo le parti di codice più importanti, demandando all'appendice il codice completo e altri extra come il manuale utente.

% 
%			CAPITOLO 5: Test
% 

\chapter{Test}
\label{chap:test}

Ogni lavoro scientifico richiede una validazione dei risultati ottenuti. Questo si può fare confrontando in modo sistematico il proprio lavoro con lavori concorrenti o misurando l'efficacia del lavoro mediante test con gli utenti. \`E fondamentale che questi test siano ripetibili, dovrete dunque fornire tutti i dettagli necessari nel testo per permettere a chi legge la tesi di replicare l'esperimento.

Progettare e condurre un test soggettivo con utenti \`e un lavoro complesso e lungo, che richiede pianificazione e competenza. Il bel libro di Lazar {\em et al.}~\cite{lazar2017methods} illustra in dettaglio i concetti principali della ricerca sperimentale e le metodologie correlate: ipotesi di ricerca, design sperimentale, analisi dei risultati sperimentali. Quelli che seguono sono alcuni consigli specifici sugli aspetti pi\`u importanti.

\section{Protocollo}

Uno degli errori più comuni è sottovalutare lo studio di un \textit{protocollo} sperimentale. Affinché i risultati dei test siano significativi è necessario non trascurare i seguenti aspetti:

\begin{itemize}
	\item eliminare distorsioni sistematiche involontarie;
	\item isolare le variabili oggetto dello studio;
	\item garantire una numerosità sufficiente del campione;
	\item confrontare gli effetti con un gruppo di controllo.
\end{itemize}

\section{Risultati}

I risultati dei test vanno presentati in modo chiaro e completo, possibilmente indicando la significatività statistica di quanto ottenuto.

\`E buona norma fornire sia i dati numerici (un esempio di come si fanno le tabelle in \LaTeX\  è visibile in Tab.\ \ref{tab:sample}), sia una rappresentazione grafica (a barre, a scatole e baffi, a violino, di dispersione, ecc.).

\`E inoltre consigliato riportare in appendice i dati grezzi completi, in modo da permettere al lettore di ripetere eventuali test statistici.

\begin{table}
	\centering
    \begin{tabular}{|l|l|l|}
	\hline
	Colonna 1 & Colonna 2 & Colonna 3 \\ \hline
	5         & 8         & 1         \\
	6         & 9         & 2         \\
	7         & 0         & 3         \\ \hline
\end{tabular}
	\caption{Tabella di esempio.}
	\label{tab:sample}
\end{table}

\section{Osservazioni}

Quando si traggono conclusioni dai dati bisogna prestare attenzione a non confondere la correlazione con un rapporto di causalità. Molto spesso accade che un test suggerisca la presenza di un fenomeno, ma non dica nulla sulla causa. In questo caso bisogna formulare delle ipotesi, calcolare le implicazioni, ed eseguire un test che valuti se e quali di queste implicazioni si verifichino. Se il nuovo test falsifica la teoria, non importa quanto questa sia elegante: è falsa. Se invece il nuovo test non falsifica la teoria, allora la si può dare per ``vera fino a prova contraria''.

Per queste ragioni è necessario esporre le proprie osservazioni in maniera cauta, senza andare oltre ciò che suggeriscono i dati. \`E certamente possibile speculare sulle cause, ma va esplicitato chiaramente, e tali speculazioni vanno supportate dalla bibliografia.

% 
%			CAPITOLO 6: Conclusioni e sviluppi futuri
% 

\chapter{Conclusioni}
\label{cap6}

\section{Conclusioni}

Nelle conclusioni si tirano le somme di quanto realizzato, facendo un riassunto stringato del lavoro svolto. In particolare vanno dichiarati punti di forza e criticità della ricerca effettuata, nonché quali aspetti dello stato dell'arte siano stati superati dal lavoro in oggetto.

\section{Sviluppi futuri}

Tra gli sviluppi futuri in genere trovano posto quelle migliorie non realizzate per mancanza di tempo, la cui necessità è emersa solo dopo i test, e che riguardano il progetto ad un livello di astrazione più alto (nel caso di tesi che si inquadrano in una linea di ricerca).


% 
%			APENDICE: materiali aggiuntivi e dimostrazioni
% 

\appendix

\chapter{Informazioni generali sull'attività di tirocinio}
Lo studente che si appresta a svolgere il lavoro di tesi generalmente incontra relatore e correlatore a inizio progetto per definire una tabella di marcia, quindi al completamento di ogni obiettivo fissato si ripresenta per un aggiornamento e un controllo. Questo è molto importante per approcciare il problema in modo organico e per evitare di illudersi di aver raggiunto un risultato sufficiente ritenendo concluso il lavoro.

Durante il lavoro è consigliato di prendere appunti su ciò che si sta facendo, in modo da avere una base per la stesura dell'elaborato.

Infine, è richiesto agli studenti di consegnare la proposta di indice prima di iniziare la stesura, per poi consegnare ogni capitolo a mano a mano che questi vengono completati. La tesi completa va consegnata per la rilettura finale a relatore e correlatore almeno una settimana prima della consegna ufficiale, per permettere un ultimo \textit{proof reading} e l'integrazione delle eventuali modifiche richieste.

Altri dettagli:
\begin{itemize}
    \item E' estremamente importante che lo studente comunichi con i relatori per aggiornamenti regolari, in particolare ogni volta che ci sono sviluppi significativi o decisioni da prendere.
    \item E' altresì fondamentale che ogni comunicazione via mail avvenga mettendo sempre tutti i relatori/correlatori tra i destinatari.
    \item Di norma non serve venire a lavorare in laboratorio, tuttavia se lo studente lo preferisce è il benvenuto. Per ottenere il badge di accesso si acceda al portale: \url{https://badge.di.unimi.it/}
\end{itemize}

\section{Aspetti Burocratici (CdL Triennali)}
Solitamente l'attività di tirocinio interno per la triennale si articola nei seguenti passaggi formali:
\begin{enumerate}
    \item Scegliere un argomento su cui lavorare: è possibile partire da una propria idea o scegliere un tema tra quelli proposti dai docenti
    \item Scegliere relatore e correlatore: il relatore deve essere uno strutturato del Dipartimento (Ricercatore, Professore associato o Professore ordinario)
    \item Definire insieme ai relatori: titolo (anche provvisorio), obbiettivi e tabella di marcia
    \item Procedere con l'apertura formale del tirocinio sulla piattaforma\\ \url{https://tirocini.di.unimi.it}
    \item Il tirocinio deve durare almeno 14 settimane. Durante questo tempo si dovrebbe svolgere il grosso del lavoro
    \item A tirocinio ultimato, se e solo se il relatore è d'accordo, si procede con la chiusura formale del tirocinio. Questa consiste da un lato nell'iscrizione all'appello apposito sul SIFA/UniMIA a carico dello studente, e dall'altro nell'inoltro della richiesta di chiusura a carico del relatore
    \item Iscrizione alla sessione di laurea sul sito del Corso di Laurea\\ \texttt{https://<nome\_cdl>.cdl.unimi.it/it/studiare/laurearsi}\\ (es. \url{https://informaticamusicale.cdl.unimi.it/it/studiare/laurearsi})
    \item Consegna di elaborato e riassunto in formato PDF-A (questo template produce già il formato richiesto) non superiore a 10MB 
    \item Discussione pubblica, che in quanto tale è aperta a spettatori esterni
    \item Festeggiamenti (si ricorda che questi devono essere svolti nel rispetto dei luoghi e delle istituzioni rappresentate)
\end{enumerate}

\section{Aspetti Pratici}
Relativamente gli aspetti pratici, ogni relatore ha un suo modus operandi per la gestione dei tesisti, quanto riportato di seguito va quindi considerato a solo titolo esemplificativo; in particolare questa è la procedura tipica seguita dagli studenti del Corso di Laurea di Informatica Musicale:
\begin{enumerate}
    \item Analisi dello stato dell'arte (Cap.~\ref{chap:stato_arte}).\\E' importante annotare tutto sin da subito usando \LaTeX, perché poi andrà scritto nell'elaborato.
    \item Definizione del problema, ``design'' del sistema, scelta degli strumenti
    \item Implementazione
    \item Definizione ed esecuzione della campagna di test e/o validazione (Cap.~\ref{chap:test})
    \item Definizione dell'indice dell'elaborato
    \item Scrittura dell'elaborato e correzione (meglio se in parallelo con gli altri passi)
    \item Consegna dell'elaborato
    \item Stesura della presentazione, correzione, suggerimenti
\end{enumerate}

\section{Gestione del Codice}
Nella maggior parte dei tirocini dei CdL in Informatica, lo studente si troverà a scrivere del codice. Si consiglia di concordare in fase preliminare il modo di condivisione del codice.
Per quanto riguarda il LIM, il codice pertinente alle tesi viene generalmente versionato su GitHub (\url{https://github.com/LIMUNIMI}).

Nel caso di utilizzo di git (con qualsiasi remoto, sia esso GitHub, GitLab, Bitbucket, \dots), si consiglia allo studente di avere una infarinatura generale prima di utilizzarlo per il proprio progetto. Un buon tutorial interattivo è \url{https://learngitbranching.js.org}, anche se include argomenti avanzati che possono essere al di là delle necessità contingenti dello studente.


\chapter{Documenti da Produrre}
I membri del LIM in generale non richiedono agli studenti di consegnare una copia cartacea del proprio lavoro. In altri casi, si invita lo studente a concordare questo punto con il proprio relatore.
In prossimità della discussione viene richiesto loro di condividere in formato digitale:
\begin{itemize}
	\item la versione finale dell'elaborato;
	\item il riassunto caricato in piattaforma;
	\item il codice sviluppato;
	\item tutti i materiali di supporto utilizzati e/o prodotti;
	\item la presentazione predisposta per la discussione.
\end{itemize}

La consegna può avvenire su supporto ottico, via chiavetta USB o attraverso i sistemi di condivisione dei file (WeTransfer, Dropbox, Google Drive, ecc.).

\section{Riassunto}
In prossimità della data di laurea, viene richiesto agli studenti di caricare in piattaforma anche il riassunto. La sua funzione è permettere alla commissione di visionare rapidamente gli argomenti trattati da ciascun elaborato.

L'estensione di tale documento non dovrebbe superare le due pagine, e spesso una pagina è più che adeguata allo scopo. Riguardo i contenuti, il riassunto riprende -- in maniera molto succinta -- lo schema dell'elaborato finale, e quindi, ad esempio, la scaletta proposta in \ref{sec:organizzazione}.

Visto lo scopo del documento, è buona prassi bilanciare il testo in modo da dedicare poche righe allo stato dell'arte, in favore delle parti originali dell'elaborato.

\section{Presentazione}
Gli studenti delle lauree triennali hanno a disposizione al massimo 10 minuti per presentare il proprio elaborato, che diventano 15 per le lauree magistrali. 

Si raccomanda di provare ripetutamente il discorso cronometrandosi e considerando i tempi morti per l'accensione del PC, l'avvio dei software, il passaggio da un'applicazione a un'altra, e via dicendo. Si segnala inoltre la possibilità di prenotare l'aula della discussione per provare in anticipo i collegamenti e verificare che immagini e audio vengano correttamente riprodotti.

Per via degli stringenti vincoli temporali, difficilmente una presentazione multimediale può superare la soglia di 15 slide. Questo valore è puramente indicativo, perché dipende dalla densità di informazioni della slide e dal tempo dedicato a ciascuna slide mentre si sta presentando.

La scelta tecnologica più comune è Microsoft PowerPoint, ma qualsiasi alternativa (Prezi, Acrobat PDF, \LaTeX, ecc.) è ammissibile. 

Gli errori più comuni da evitare sono:
\begin{itemize}
	\item slide dal contenuto testuale troppo ricco;
	\item slide che verranno lette parola per parola in fase di presentazione.
\end{itemize}

Entrambi i problemi si risolvono impostando il testo in maniera estremamente schematica, tendenzialmente per elenchi puntati. Si presti attenzione, però, a usare una forma omogenea. 
Ad esempio, sarebbe poco elegante il seguente elenco, che include aggettivi, sostantivi e frasi di senso compiuto, e non fa un uso corretto di iniziali maiuscole:
\begin{itemize}
	\item applicativo esteticamente molto gradevole;
	\item C'è la possibilità di interagire con altri utenti;
	\item l'utente può esportare file in formato XML.
\end{itemize}

Una possibile revisione, basata solo sui sostantivi, è: 
\begin{itemize}
	\item qualità estetica;
	\item interazione tra utenti;
	\item supporto dell'XML.
\end{itemize}

Un altro consiglio per rendere la presentazione cognitivamente più gestibile da parte del pubblico è quello di aggiungere in calce ad ogni slide un indicatore dello stato di avanzamento (ad esempio ``Slide 3 di 15''). In questo modo, in caso di esaurimento del tempo concesso la commissione saprà se dare modo di concludere (``Slide 14 di 15'') o se fermare il candidato (``Slide 14 di 138''). In assenza di un indicatore la commissione tende a presupporre il secondo caso.

Per quanto riguarda gli argomenti da trattare durante la discussione, è opportuno esporre brevemente lo stato dell'arte per concentrarsi sul proprio lavoro, sui test e sui risultati. Una tecnica per capire quanto tempo dedicare a ogni argomento consiste nel dare un peso a ciascuna parte di cui si vuole parlare, quindi assegnare un numero di minuti proporzionale a detto peso in modo da raggiungere il tempo massimo a disposizione.

Video dimostrativi e \textit{live demo} sono gradite, ma è sconsigliabile dedicare troppo tempo a questi contributi, a meno che al contenuto multimediale si sovrapponga una spiegazione da parte del candidato.


%
%			BIBLIOGRAFIA
%

% Si può specificare a che livello della TOC deve essere la bibliografia.
% Il default è 'chapter', per 'part' usare
% \beforebibliography[part]
\beforebibliography
\bibliographystyle{unsrt}
\bibliography{bibliografia}

% Pagina di chiusura
\closingpage

\end{document}
