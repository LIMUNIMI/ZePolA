\documentclass[12pt]{report}
\usepackage[backend=biber, style=apa]{biblatex}
\addbibresource{bibliografia.bib}
\linespread{1.3}

% --- PREAMBOLO ---------------------------------------------------------------
% Inserire qui eventuali package da includere o
% definizioni di comandi personalizzati

% Selezione lingua
\usepackage[italian]{babel}

\usepackage{tesi}
% Puoi usare il font di default di LaTeX con la relativa opzione del package
% \usepackage[defaultfont]{tesi}
% Esiste anche un'opzione per il formato 17x24 per le tesi di dottorato
% \usepackage[phd]{tesi}

% In caso il copia-incolla del PDF generato perda gli spazi,
% provare a decommentare la seguente riga
% \pdfinterwordspaceon

% !!! INFORMAZIONI SULLA TESI DA COMPILARE !!!

%   UNIVERSITA' E CORSO DI LAUREA:
\university{Università degli Studi di Milano}
\unilogo{immagini/unimi}
\faculty{Facoltà di Scienze e Tecnologie}
\department{Dipartimento di Informatica\\Giovanni Degli Antoni}
\cdl{Corso di Laurea Triennale in Informatica Musicale}

%   TITOLO TESI:
\title{Equalizzatore parametrico con GUI per poli e zeri: uno strumento didattico}
% Questo comando (opzionale) sovrascrive \title per quanto riguarda la copertina
% Può essere usato per stampare caratteri speciali, tenendo i metadati puliti
\printedtitle{Equalizzatore parametrico con GUI per poli e zeri: uno strumento didattico}

%   AUTORE:
\author{Andrea Casati}
\matricola{963119}
% "Elaborato Finale" per i CdL triennali
% "Tesi di Laurea" per i CdL magistrali
\typeofthesis{Elaborato Finale}

%   RELATORE E CORRELATORE:
\relatore{Prof. Giorgio Presti}
\correlatore{Prof. Marco Tiraboschi}

%   LABORATORIO:
% Questa sezione crea una pagina di chiusura della tesi con
% il logo dell'ente/laboratorio presso cui si è svolto il tirocinio.
% Più afferenze/url/loghi sono supportate,
% e la frase può essere personalizzata.
% Qui trovate alcuni predefiniti del nostro dipartimento
% \adaptlab
% \aislab
% \anacletolab
% \bisplab
% \connetslab
% \everywarelab
% \falselab
% \iebilab
% \islab
% \lailalab
% \lalalab
% \lawlab
% \laserlab
% \limlab
% \mipslab
% \optlab
% \phuselab
% \ponglab
% \sesarlab
% \spdplab

% Esempio di personalizzazione della pagina di chiusura
% (non consegnate con questo esempio!)
% (da commentare in caso sia sufficiente una delle macro precedenti)
% \lab{Laboratorio di Ricerca}
% \lab[in collaborazione con l']{Azienda Specifica}
% \laburl{https://di.unimi.it/it/ricerca/risorse-e-luoghi-della-ricerca/laboratori-di-ricerca}
% \lablogo{immagini/redqmark}

%   ANNO ACCADEMICO:
% \the\year inserisce l'anno corrente
% per specificare manualmente un anno accademico
% NON inserire nel formato 1970-1971, ma
% inserire solo 1970
\academicyear{\the\year}

%   INDICI:
% elenco delle figure (facoltativo)
% \figurespagetrue
% elenco delle tabelle (facoltativo)
% \tablespagetrue
% prefazioni nell'indice (facoltativo)
\prefaceintoctrue
% indice nell'indice (facoltativo)
\tocintoctrue
% --- FINE PREAMBOLO ----------------------------------------------------------

\begin{document}

% Creazione automatica della copertina
% Centra la copertina nel foglio: usa questo comando per la copertina esterna
% \makecenteredfrontpage
% Copertina allineata alle altre pagine: usa questo comando per la copertina interna
\makefrontpage

%
%            PAGINA DI DEDICA E/O CITAZIONE
%            facoltativa, questa è l'unica cosa che dovete formattare a mano, un po' come vi pare
%

{\raggedleft \large \sl Questo lavoro \`{e} dedicato ai miei genitori\\}

\clearpage
\beforepreface

%
%            PREFAZIONE (facoltativa)
%

% \prefacesection{Prefazione}
% Le prefazioni non sono molto comuni, tuttavia a volte capita che qualcuno voglia dire qualcosa che esuli dal lavoro in sé (come un meta-commento sull'elaborato), o voglia fornire informazioni riguardanti l'eventuale progetto entro cui la tesi si colloca (in questo caso è probabile che sia il relatore a scrivere questa parte).

%
%            RINGRAZIAMENTI (facoltativi)
%

\prefacesection{Ringraziamenti}
Questa sezione, facoltativa, contiene i ringraziamenti.

%
%            Creazione automatica dell'indice
%

\afterpreface

%
%            CAPITOLO 1: Introduzione o Abstract
%

\chapter{Introduzione}
\label{cap:introduzione}

\section{Il template}

\section{I contenuti}

\section{Organizzazione della tesi}

\section{Stile e forma}

\chapter{Filtri digitali: una panoramica teorica introduttiva}
\label{cap:panoramica_introduttiva}

%
%            CAPITOLO 3: Stato dell'arte
%

\chapter{Stato dell'arte}
\label{chap:stato_arte}
La ricerca e lo sviluppo nel filtraggio digitale ebbero inizio negli anni Sessanta, quando l’introduzione del filtraggio digitale ricorsivo aprì la strada alla simulazione di filtri analogici nel dominio digitale.

I primi studi sul filtraggio digitale sono stati incentivati dalla necessità di creare tecnologie capaci di simulare dispositivi per la modifica della larghezza di banda della voce. L'uso dei filtri analogici per il trattamento dei segnali vocali, infatti, comportava varie difficoltà, tra cui costi elevati, peso e ingombro dell'hardware.

I primi approcci al filtraggio digitale, in particolare al filtraggio ricorsivo, si sono basati sul concetto di convoluzione: l’output di un filtro digitale è legato al suo input tramite l’integrale di convoluzione, approssimabile in digitale come somma di convoluzione del segnale d’ingresso e della risposta all’impulso campionati. La principale difficoltà, oltre alla limitata capacità di calcolo degli anni Sessanta rispetto a oggi, era la gestione della risposta all'impulso infinita dei filtri ricorsivi.

L’intuizione fu quella di applicare lo stesso principio utilizzato nei filtri analogici: l’uso di una \textit{memoria}. I filtri analogici, infatti, generano ricorsività attraverso circuiti di retroazione, in cui l’uscita viene riproposta in ingresso.

Lo studio delle equazioni alle differenze nei filtri analogici rese quindi naturale l’applicazione delle stesse in ambito digitale, tramite l'utilizzo della trasformata Z, permettendo di implementare queste equazioni come un vero e proprio software applicativo.

I passaggi successivi riguardarono la mappatura del piano \( s \) (tramite la trasformata di Laplace) nel piano \( z \) (tramite la trasformata \( Z \)) per il calcolo di poli e zeri del filtro e la loro effettiva implementazione. Se nei filtri analogici poli e zeri vengono realizzati attraverso l’uso di resistori, condensatori e induttori, in digitale si è subito dimostrato possibile realizzarli esclusivamente mediante calcoli matematici, con particolare attenzione alla precisione numerica.


Fin dall'inizio, l'elaborazione numerica digitale dei segnali ha mostrato vantaggi rispetto alla controparte analogica: mentre l’elaborazione analogica può dimostrarsi effimera, quella digitale permette il salvataggio dell'ingresso, dello stato di un filtro e dell'uscita.

Nel corso degli anni, in particolare dagli anni Settanta (periodo delle prime applicazioni pratiche dei filtri digitali) fino a oggi, le proprietà e gli utilizzi dei filtri digitali non ricorsivi FIR (con risposta all’impulso finita) e ricorsivi IIR (con risposta all’impulso infinita) sono cambiati significativamente.

Negli anni successivi, i filtri digitali sono stati impiegati anche per altri scopi, oltre al filtraggio selettivo, come il matched-filtering e adaptive-filtering (utilizzato nelle telecomunicazioni). In questi contesti, i filtri FIR si sono rivelati particolarmente efficaci. Durante gli anni Ottanta, quindi, i filtri digitali FIR sono diventati preferibili per questi diversi utilizzi.

Con il progresso degli anni Novanta e la possibilità di rendere stabili i filtri digitali IIR e a fase lineare, questi ultimi hanno ritrovato un ruolo importante nelle applicazioni digitali.

Oggi i filtri digitali ricorsivi sono considerati un metodo efficace per approssimare una risposta all’impulso infinita. In teoria infatti, la risposta all’impulso di un filtro ricorsivo è infinita nel tempo, ma in pratica essa decresce esponenzialmente verso lo zero. Nel dominio digitale, quindi, l’errore di approssimazione diventa trascurabile già dopo poche centinaia di campioni \parencite{rader2006rise}.

\section{Equalizzazione audio digitale}
L’equalizzazione è una tecnica di elaborazione del segnale utilizzata per modificare la risposta in frequenza di un segnale audio, enfatizzando o attenuando determinate bande di frequenza. Questa pratica ha radici storiche profonde e ha subito una continua evoluzione grazie ai progressi della tecnologia audio e del trattamento digitale dei segnali. Inizialmente sviluppata per esigenze di comunicazione e successivamente affinata in contesti musicali e cinematografici, l’equalizzazione è diventata una componente essenziale nell’ingegneria audio moderna. Con l’avvento dei filtri digitali, è ora possibile modellare il suono con una precisione e una flessibilità impensabili con le sole tecnologie analogiche, permettendo non solo la riproduzione fedele ma anche la sperimentazione sonora in contesti di mixaggio, registrazione e produzione.
\subsection{Equalizzazione nell'elaborazione di segnali}
Il concetto di manipolazione delle frequenze nei segnali risale all’avvento del telegrafo e delle comunicazioni degli anni Settanta dell'Ottocento. I primi usi riguardano il filtraggio dei segnali del telegrafo armonico, in cui ciascun tasto, una volta premuto, attivava la vibrazione di un’ancia elettromeccanica associata a una specifica frequenza. Lo strumento ricevente, per decodificare correttamente il segnale, doveva sintonizzare i propri tasti sulle frequenze corrispondenti a quelle inviate.

Il termine equalizzazione ha origine nell'ingegneria telefonica dei primi tempi, settore in cui era necessario correggere i segnali telefonici degradati, soprattutto a causa della perdita delle alte frequenze nei lunghi tragitti, per ripristinare il segnale originale all'arrivo a destinazione.

I primi prototipi di equalizzatori variabili, dotati di controlli per modificare la risposta in frequenza, nacquero con il cinema sonoro. I controlli di tono, guadagno e attenuazione delle alte e basse frequenze furono poi introdotti con il grammofono tra gli anni Quaranta e Cinquanta del Novecento.

Già negli anni Cinquanta del Novecento i circuiti di equalizzazione divennero degli standard nella produzione musicale, specialmente nella produzione e riproduzione di dischi LP.

Nella decade Cinquanta-Sessanta del ventesimo secolo, gli equalizzatori acquisirono popolarità anche nella post-produzione audio musicale: furono sviluppati nuovi modelli e strumenti, dagli equalizzatori grafici fino ai primi equalizzatori parametrici degli anni Settanta.

L'avvento dell'equalizzazione digitale avvenne a cavallo tra gli anni Settanta e Ottanta del secolo scorso: il primo equalizzatore parametrico digitale basato su DSP in commercio fu il DEQ7 della Yamaha, uscito sul mercato nel 1987 \parencite{reiss2016all}.

L'equalizzazione è quindi divenuto un processo cruciale nell'elaborazione audio: consente infatti di enfatizzare o attenuare specifiche bande di frequenza, migliorando la chiarezza, l'intelligibilità o il carattere sonoro di un segnale audio. Dagli studi di registrazione alle applicazioni di live sound e produzione musicale, gli equalizzatori digitali sono utilizzati per adattare il suono a contesti acustici diversi, correggere problematiche sonore o migliorare il contenuto audio in base ai gusti estetici e alle esigenze del mixaggio.

Con l’avvento dei filtri digitali, l’equalizzazione ha raggiunto un alto livello di flessibilità, permettendo sia la riproduzione e simulazione di strumenti analogici, sia la creazione di strumenti e software innovativi che superano le
più classiche e tradizionali implementazioni.

Gli equalizzatori digitali si dividono in categorie:
\begin{itemize}
    \item Equalizzatori grafici: strumenti per la manipolazione della risposta in frequenza, che offrono un controllo di volume (attenuazione e guadagno) su bande di frequenza specifiche, stabilite in fase di progettazione (solitamente rispettando le frequenze presenti nelle tabelle standard ISO). Ogni banda è regolabile tramite cursori o controlli dedicati, che agiscono su filtri passa-banda per modellare lo spettro audio secondo le preferenze dell'utente \parencite{liski2017quest}.
    \item Equalizzatori parametrici: strumenti per la manipolazione della risposta in frequenza che offrono un controllo adattabile su ogni banda di frequenza. Consentono di regolare non solo il guadagno o l'attenuazione, ma anche la frequenza centrale e la larghezza di banda (o Q factor) di ogni singolo filtro. Questi parametri permettono una modellazione precisa dello spettro audio, adattandosi meglio alle esigenze specifiche dell'utente \parencite{massenburg1972parametric}.
    \item Equalizzatori dinamici: strumenti per la manipolazione della risposta in frequenza che offrono un controllo basato sul principio degli equalizzatori parametrici. Tuttavia, a differenza di questi ultimi, l'attenuazione o il guadagno di specifiche bande dipende dal superamento di una soglia di volume nello spettro della stessa banda di frequenza. La risposta del filtro cambia quindi a seconda della variazione dinamica dello spettro del segnale \parencite{martignon2023dynamic}.
\end{itemize}
\subsection{Metodi di progettazione di equalizzatori digitali: catena di filtri biquadratici}
Gran parte delle applicazioni di elaborazione dei segnali riguarda l'attenuazione o l'accentuazione di una porzione specifica dello spettro del segnale.
Un equalizzatore parametrico, come già accennato, è caratterizzato da frequenza centrale \(\omega_c\) e larghezza di banda (intervallo di frequenze attorno alla frequenza centrale su cui il filtro ha effetto) detta bandwidth \textit{B}, da cui deriva il Q-factor \textit{Q} calcolato come:
\begin{equation}
    \textit{Q}=\frac{\omega_c}{\textit{B}}
    \label{eq:q-factor}
\end{equation}
In ambito audio, questo obiettivo viene spesso raggiunto attraverso l’uso di filtri biquadratici, che consentono di attenuare o amplificare una specifica banda di frequenze attorno a una frequenza centrale \parencite{reiss2010design}.
In gran parte delle applicazioni audio infatti è sufficiente l'utilizzo di filtri del secondo ordine in quanto sono necessari valori di \textit{Q}-factor moderati. Nelle implementazioni digitali in generale possono essere utilizzati sia filtri FIR che filtri IIR. Sebbene i filtri FIR siano intrinsecamente stabili e permettano una fase lineare, ottenere una risposta in frequenza desiderata può richiedere ordini elevati, aumentando di conseguenza la capacità computazionale necessaria, il che rende questo tipo di filtri inadatti per performance live e processamento real-time.
I filtri IIR, al contrario, permettono di mantenere un ordine del filtro più basso, richiedendo così una minore capacità computazionale, ma spesso presentano una fase non lineare e possono essere instabili. Inoltre, con questo tipo di filtri è necessario prestare particolare attenzione alla precisione numerica \parencite{shpak1991analytical}.

Nel dominio digitale, è possibile ottenere qualsiasi risposta in frequenza attraverso una cascata di filtri biquadratici disposti in serie \parencite{reiss2010design}. Per un equalizzatore parametrico, nel quale l'utente ha la possibilità di modificare il valore della frequenza centrale \(\omega_c\), la larghezza di banda \textit{B} (e quindi il fattore \textit{Q}) e il guadagno (o attenuazione), la scelta di filtri \textit{"semplici"} in cascata è da preferire, rispetto a una configurazione in parallelo (come nel caso di equalizzatori grafici) in quanto la risposta in ampiezza del filtro è data dal prodotto delle risposte in ampiezza dei singoli filtri \parencite{shpak1991analytical}.

L'approccio più utilizzato per la creazione di filtri digitali è partendo dal prototipo analogico da realizzare per poi mappare, tramite la trasformazione bilineare, l'asse delle frequenze analogiche \([0,1)\) sull'asse delle frequenze digitali \([0,\frac{\omega_s}{2}]\), dove \(\omega_s\) è la frequenza di campionamento in radianti.
La progettazione del filtro analogico prototipo e le conseguenti trasformazioni consentono di ottenere, a partire dalla frequenza centrale \(\omega_c\), dal guadagno (o attenuazione) e dalla larghezza di banda \textit{B}, i valori dei coefficienti dei filtri biquadratici digitali in catena \parencite{reiss2010design}.

La funzione di trasferimento di un filtro biquadratico nel dominio \( s \) analogico è:
\begin{equation}
    H(s)=g_\infty\frac{s^2 + \frac{\omega_z}{Q_z}s + \omega_z^2}{s^2 + \frac{\omega_p}{Q_p}s + \omega_p^2}
    \label{eq:analog_biquad}
\end{equation}

definita da:
\begin{itemize}
    \item \(g_\infty: \) guadagno complessivo dato dal valore di \(\displaystyle \lim_{\omega \to \infty} H(j\omega) \)
    \item una frequenza per lo zero \(\omega_z\) e il fattore \(Q_z\) per la larghezza di banda
    \item una frequenza per il polo \(\omega_p\) e il fattore \(Q_p\) per la larghezza di banda
\end{itemize}
\parencite{christensen2003generalization}

Nel dominio digitale la funzione di trasferimento di un filtro biquadratico è la seguente:
\begin{equation}
    H(z)=c\frac{(z-z_1)(z-z_2)}{(z-p_1)(z-p_2)} = c\frac{1-(z_1+z_2)z^{-1} + z_1z_2z^{-2}}{1-(p_1+p_2)z^{-1} + p_1p_2z^{-2}}
    \label{eq:non_simmetric_digital_biquad}
\end{equation}
\parencite{reiss2010design}

Considerando \( H(z) \) la funzione di trasferimento del filtro e \( z = e^{j2\pi \phi} \), dove \( \phi \) è la frequenza normalizzata rispetto alla frequenza di campionamento \( f_s \), per mantenere la simmetria hermitiana della funzione di trasferimento (\( H(\phi) = H(-\phi) \)) è necessario che i due zeri e i due poli del filtro biquadratico siano coppie di complessi coniugati. Mantenere la simmetria hermitiana della funzione di trasferimento di un filtro digitale consente di ottenere un'uscita reale quando il segnale in ingresso è reale, una condizione essenziale nell’elaborazione di segnali audio \parencite{pedersini2020elementi}.

La funzione di trasferimento di un filtro biquadratico con uscita reale è quindi:
\begin{equation}
    H(z)=\frac{(1-z_1z^{-1})(1-\overline{z_1}z^{-1})}{(1-p_1z^{-1})(1-\overline{p_1}z^{-1})}
    \label{eq:simmetric_digital_biquad}
\end{equation}
dove:
\begin{itemize}
    \item \(z_1\) e \(\overline{z_1}\) sono gli zeri coniugati del filtro e \(z_1=\rho_{z_1} e^{-j2\pi \phi_{z_1}} \)
    \item \(p_1\) e \(\overline{p_1}\) sono i poli coniugati del filtro e \(p_1=\rho_{p_1} e^{-j2\pi \phi_{p_1}} \)
\end{itemize}
In questo caso il guadagno del filtro è pari a 1.
Nel caso di zeri e poli coniugati la funzione di trasferimento \(H(z)\) è esprimibile come:
\begin{equation}
    H(z) = \frac{1 - 2\Re[z_1]z^{-1}+|z_1|^2z^{-2}}{1 - 2\Re[p_1]z^{-1}+|p_1|^2z^{-2}}=\frac{1 - 2\cos(2\pi \phi_{z_1})z^{-1}+\rho_{z_1}^2z^{-2}}{1 - 2\cos(2\pi \phi_{p_1})z^{-1}+\rho_{p_1}^2z^{-2}}
    \label{eq: digital_biquad_H}
\end{equation}
A partire dalla funzione di trasferimento del filtro è possibile ottenere, mediante l'antitrasformata \(Z\) l'equazione alle differenze di un filtro biquadratico, che descrive l'uscita dal filtro a partire dell'ingresso del campione \(n\).
\begin{equation}
    y(n)=b_0x(n)+b_1x(n-1)+b_2x(n-2) - a_1y(n-1) - a_2y(n-2)
    \label{eq:eq_differenze_generale}
\end{equation}

Per lo sviluppo dell'antitrasformata:
\begin{itemize}
    \item \(b_0 = 1\)
    \item \(b_1=\Re[z_1]=- 2\cos(2\pi \phi_{z_1}) \)
    \item \(b_2=|z_1|^2= \rho_{z_1}^2\)
    \item \(a_1=\Re[p_1]=- 2\cos(2\pi \phi_{p_1}) \)
    \item \(a_2=|p_1|^2= \rho_{p_1}^2\)
\end{itemize}
L'equazione alle differenze diventa quindi:
\begin{equation}
    \mathcal{Z}^{-1}\{H(z)\} =  y(n)=b_0x(n)+b_1x(n-1)+b_2x(n-2) - a_1y(n-1) - a_2y(n-2)
    \label{eq:biquad_eq_differenze}
\end{equation}
\begin{equation}
    y(n) = x(n) - 2\cos(2\pi \phi_{z_1})x(n-1) + \rho_{z_1}^2x(n-2) + 2\cos(2\pi \phi_{p_1})y(n-1) + \rho_{z_1}^2y(n-2)
    \label{eq:biquad_eq_differenze_risolta}
\end{equation}

È possibile inoltre applicare un guadagno all'intero filtro, in modo da compensare la differenza di ampiezza tra l'ingresso e l'uscita semplicemente moltiplicando per un valore (di guadagno lineare) l'ingresso \(x(n)\). Si tratta quindi di assegnare un valore diverso da 1 al coefficiente \(b_0\).

Un filtro digitale biquadratico è, in sintesi, un filtro del secondo ordine versatile e particolarmente adatto, grazie alla sua semplicità di implementazione, per la composizione in cascata con altri filtri biquadratici per formare un equalizzatore parametrico.

\section{Prototipi di filtri analogici per il design di filtri digitali}
Per realizzare un filtro digitale è necessario partire dal prototipo analogico, ovvero il corrispondente filtro nel dominio analogico sul piano \(s\). I passaggi da effettuare sono i seguenti:
\begin{enumerate}
    \item Conversione delle specifiche del filtro digitale da realizzare in specifiche equivalenti per il prototipo analogico lowpass
    \item Design del prototipo lowpass analogico
    \item Conversione della funzione di trasferimento \(H_a(s)\) del prototipo nell'equivalente funzione di trasferimento digitale \(H(z)\) mediante una trasformazione bilineare che mappa i valori del piano \(s\) nei valori del piano \(z\)
\end{enumerate}


%
%            CAPITOLO 3: Il lavoro svolto
%
 

%
%            BIBLIOGRAFIA
%

% Si può specificare a che livello della TOC deve essere la bibliografia.
% Il default è 'chapter', per 'part' usare
% \beforebibliography[part]
\printbibliography

% Pagina di chiusura
\closingpage

\end{document}

